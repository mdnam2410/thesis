\subsection{Dataset}
\label{subsec:dataset}
In our study, we utilize two distinct datasets for the purpose of conducting a chest X-ray image classification task. By employing multiple datasets, we aim to enhance the generalizability of our findings.


\subsubsection{Chest X-Ray Images with Pneumothorax Masks dataset}
\label{subsubsec:pneumothorax}
This dataset contains the stage 1 train and test data extracted from the Kaggle SIIM-ACR Pneumothorax Segmentation competition \cite{pneumothorax}. This dataset encompasses a collection of medical images specifically curated for investigating and addressing the challenges associated with pneumothorax, a potentially life-threatening condition characterized by the presence of air in the pleural cavity, leading to lung collapse. This dataset comprises a total of 12,047 chest X-ray images, encompassing 10,675 samples designated for training and an additional 1,372 samples for testing. To ensure the reliability and generalizability of our models, we further partition the training samples into training and validation subsets using a ratio of 9:1, respectively. This division facilitates the assessment of model performance on unseen data and enables the selection of optimal hyper-parameters.

\subsubsection{Chest X-Ray Images (Pneumonia)}.
\label{subsubsec:pneumonia}
This data includes a total of 5,863 JPEG images of X-Ray scans, encompassing 5216 samples for training and 624 samples for testing, which categorized into two classes: pneumonia and normal \cite{penumonia}. The chest X-ray images (anterior-posterior) were obtained from retrospective cohorts of pediatric patients aged one to five years old, specifically from Guangzhou Women and Children’s Medical Center in Guangzhou. These images were part of the routine clinical care provided to the patients. To ensure the quality of the chest X-ray images used for analysis, a preliminary screening process was conducted. This involved removing scans that were of low quality or deemed unreadable. Subsequently, two experienced physicians reviewed and graded the diagnoses for the remaining images before they were considered suitable for training the AI system. To account for any potential grading errors, a third expert also examined the evaluation set.
