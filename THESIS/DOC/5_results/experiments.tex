\section{Experiments}
To evaluate post-hoc XAI methods, our experiment consists of two stages. In the first stage, we train black boxes on the X-ray image classification task. In the second stage, we apply XAI methods to the trained models to gain insights into their decision-making process and interpret the underlying factors that contribute to their predictions. By analyzing the explanations provided by the XAI methods, we aim to understand the model's behavior and assess its transparency and interpretability.



The development environment in which the experiments were conducted are summarized in table \ref{table:env}.


\subsection{Dataset}
\label{subsec:dataset}
In our study, we utilize two distinct datasets for the purpose of conducting a chest X-ray image classification task. By employing multiple datasets, we aim to enhance the generalizability of our findings.


\subsubsection{Chest X-Ray Images with Pneumothorax Masks dataset}
\label{subsubsec:pneumothorax}
This dataset contains the stage 1 train and test data extracted from the Kaggle SIIM-ACR Pneumothorax Segmentation competition \cite{pneumothorax}. This dataset encompasses a collection of medical images specifically curated for investigating and addressing the challenges associated with pneumothorax, a potentially life-threatening condition characterized by the presence of air in the pleural cavity, leading to lung collapse. This dataset comprises a total of 12,047 chest X-ray images, encompassing 10,675 samples designated for training and an additional 1,372 samples for testing. To ensure the reliability and generalizability of our models, we further partition the training samples into training and validation subsets using a ratio of 9:1, respectively. This division facilitates the assessment of model performance on unseen data and enables the selection of optimal hyper-parameters.

\subsubsection{Chest X-Ray Images (Pneumonia)}.
\label{subsubsec:pneumonia}
This data includes a total of 5,863 JPEG images of X-Ray scans, encompassing 5216 samples for training and 624 samples for testing, which categorized into two classes: pneumonia and normal \cite{penumonia}. The chest X-ray images (anterior-posterior) were obtained from retrospective cohorts of pediatric patients aged one to five years old, specifically from Guangzhou Women and Children’s Medical Center in Guangzhou. These images were part of the routine clinical care provided to the patients. To ensure the quality of the chest X-ray images used for analysis, a preliminary screening process was conducted. This involved removing scans that were of low quality or deemed unreadable. Subsequently, two experienced physicians reviewed and graded the diagnoses for the remaining images before they were considered suitable for training the AI system. To account for any potential grading errors, a third expert also examined the evaluation set.


\begin{table}[h!]
\caption{Experiment environments and requirements.}\label{table:env}
\centering
\begin{tabular}{ll}
\hline
CPU   & Intel(R) Xeon(R) CPU @ 2.20GHz \\
\hline
RAM                         &1$\times $13GB \\
\hline
GPU (number and type)                         & NVIDIA TESLA P100 GPUs 16GB\\
\hline
CUDA version                  & 11.6\\                          \hline
Programming language                 & Python 3.10\\ 
\hline
\end{tabular}
\end{table}

% \vspace{-0.5cm}

\subsection{Training Stage}
In the training stage, we trained the InceptionV3 \cite{inceptionv3} and Resnet101 \cite{resnet101} models using popular tools such as pytorch, torchvision, scikit-learn, and pandas. Both models were initially pre-trained on the ImageNet dataset \cite{imageNet}, which provided them with a strong foundation in recognizing and classifying various features. To adapt these models to our specific task of pneumothorax classification, we further trained them using the Chest X-ray Images with Pneumothorax Masks dataset \cite{pneumothorax}. Before training the models, we performed preprocessing on the chest X-ray images. This involved normalization, where the image is transformed such that its mean $\mu$ is (0.485, 0.456, 0.406) and its standard deviation $\sigma$ is (0.229, 0.224, 0.225). This normalization step helps to standardize the input data and improve the training process. Additionally, we resized the images to a specific dimension with length and width in pixels, respectively: (299, 299) for InceptionV3 and (244, 244) for ResNet101. This resizing ensured that the images were compatible with the input requirements of each model. The models were trained using the classification task, aiming to categorize whether a given chest X-ray image exhibited signs of pneumothorax or not. For this binary classification task, we employed the cross-entropy loss function, which is commonly used for training classification models. The optimization of the models was carried out using the Adam optimizer, a popular choice known for its efficiency in handling complex models. The training process spanned 30 epochs, allowing the models to iteratively learn and improve their performance. Throughout the training phase, we monitored the models' progress and selected the best-performing state based on their performance on the validation set.

\subsection{Analysis Stage}
In the XAI analysis stage, we employed XAI methods provided by the Captum framework to generate explanations for the predictions made by our trained models. To measure the disagreement between these explanations, we implement some metrics introduced in \ref{subsec:metrics} by using popular python libraries such as numpy, torch, and skimage to perform the necessary calculations and computations. Once the metrics were computed, we visualized the results to gain a better understanding of the differences and similarities among the XAI methods by using seaborn library.